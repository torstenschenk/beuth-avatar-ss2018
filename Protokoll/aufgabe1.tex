\section*{Aufgabe 1: }

Eine Funktion $f : \mathbb{R}^3 \to \mathbb{R}$ sollte folgende Kriterien erfüllen, um eine implizite Fläche zu beschreiben.\\

Um Innen von Außen unterscheiden zu können: \\
$f(P_1) > 0$ und $f(P_2) < 0$ \hspace{5pt} $\Rightarrow$ \hspace{5pt} Auf jedem Weg zwischen $P_1$ und $P_2$ gibt es eine Nullstelle.\\

Um wirklich eine Fläche und kein Volumen von Nullstellen zu erhalten:\\
Es gibt kein $P \in \mathbb{R}^3$ und $r \in \mathbb{R}$ mit $r > 0$, sodass für alle $x \in \mathbb{R}^3$ mit $\lVert P - x \rVert < r$ gilt $f(x) = 0$.

