\section{Gstreamer}

%%%%%%%%%%%%%%%%%%%%%%%%%%%%%%%%%%%%%%%%%%%%%%%%%%%%%%%%%%%%%%%%%%%%%%%%%%
\subsection{Installation}
Mittels Paketmanager APT (Debian/Ubuntu), yum (Fedora/Centos) or homebrew (Mac) 
fehlende Pakete installieren, am Beispiel Ubuntu (Rapian Debian):
\begin{verbatim}
  sudo apt update 
  sudo apt install libgstreamer1.0-0 gstreamer1.0-plugins-base \
    gstreamer1.0-plugins-good gstreamer1.0-plugins-bad \
    gstreamer1.0-plugins-ugly gstreamer1.0-libav gstreamer1.0-doc \
    gstreamer1.0-tools
  sudo apt install libgstreamer1.0-dev
\end{verbatim}

%%%%%%%%%%%%%%%%%%%%%%%%%%%%%%%%%%%%%%%%%%%%%%%%%%%%%%%%%%%%%%%%%%%%%%%%%%
\textbf{I/O Elemente}
Ein Gstreamer Befehl ist aus einer Kette von Plugins aufgebaut. Die Verkettung der Befehle startet mit ein oder mehreren \textbf{src} Plugins und endet mit einem oder mehreren \textbf{sink} Plugins. Es folgt eine List mit src und sink Plugins.\\

\textbf{Beschreibung der verwendeten Eingangs oder SRC-Elemente}
\begin{itemize}
\item v4l2src – stream from a camera device on a linux system, e.g. device=/dev/video0;
\item audiotestsrc – used to do test streams with audio;
\item videotestsrc – used to do test streams with video, you may specify a pattern=<num>;
\item fakesrc – another option for testing by feeding in an empty stream;
\item filesrc – stream from a file, specifiy location=<filepath>;
\item ximagesrc – capture screen.
\end{itemize}

\textbf{Ausgabe oder SINK Elemente}
\begin{itemize}
\item filesink – save stream to a file, specify location=<filepath>;
\item  autoaudiosink – play audio on an automatically detected device;
\item  autovideosink – play video on an automatically detected display utility and device;
\item  fakesink – do not play stream, just finish;
\item  udpsink – stream result over UDP, specify host=<IP of the target server> and port=<number>;
\item rtmpsink – stream result over RTMP, specify host=<IP of the target server> and port=<number>.
\end{itemize}

%%%%%%%%%%%%%%%%%%%%%%%%%%%%%%%%%%%%%%%%%%%%%%%%%%%%%%%%%%%%%%%%%%%%%%%%%%
\textbf{Kodierungselemente}
Encoding a raw data stream and decoding encoded data into raw data is a big part of using gstreamer and dealing with media in general. Audio data can be encoded to various formats such as MP3, AAC, Vorbis and Opus. Video can be encoded among others as JPEG 2000, H.264, H.265, MPEG-2, VP8, VP9 and Theora. Gstreamer offers the possibilities to encode/decode in these formats through is bundle of plugins. These plugin bundles are the base, good, bad and ugly plugins.\\

\textbf{Audio}
\begin{itemize}
\item mp3 – lamemp3enc, avenc\_mp3 | mad, mpg123audiodec, avdec\_mp3; 
\item aac – voaccenc, faac, avenc\_aac | faad, aacparse, avdec\_aac;
\item vorbis – vorbisenc | vorbisdec, vorbisparse;
\item opus – opusenc, avenc\_opus | opusdec, avdec\_opus.
\end{itemize}

\textbf{Video}
\begin{itemize}
\item h.264 – x264enc, avh264\_enc |  h264parse, mpeg4videoparse, avdec\_h264;
\item mpeg2 -mpeg2enc, avenc\_mpeg2video | mpeg2dec, avdec\_mpeg2video;
\item jpeg2000 – no inter-frame coding, low latency; avenc\_jpeg2000 | avdec\_jpeg2000;
\item vp8 – vp8enc, avenc\_vp8 | vp8dec, avdec\_vp8;
\item vp9 – vp9enc, avenc\_vp9 | vp9dec, avdec\_vp9;
\item theora -theoraenc | theoradec, theoraparse.
\end{itemize}

%%%%%%%%%%%%%%%%%%%%%%%%%%%%%%%%%%%%%%%%%%%%%%%%%%%%%%%%%%%%%%%%%%%%%%%%%%
\textbf{Möglichkeiten oder ‘caps’ Elemente}

Siehe web tutorial, sehr gute Beschreibung,,, wo man die receiver caps herbekommt....\\

Capabilities (short: caps) describe the type of data that is streamed between two pads (elements), or the one that a pad (template) supports. Capsfilters or \textbf{caps} do not modify data as such, but can enforce limitations on the data format. They ensure compatibility between elements. If for example a media stream is available in several formats, they can specify just one which is understandable by the next element in the pipeline. Capabilities can be very complex and specify all types of characteristics although that is often not required. Most often we should specify the type of encoding of the stream we receive or send.\\

Beispiele\\
Legt kein Encoding für das Video fest \textbf{RAW} und bestimmt die Breite x Höhe einer Framegröße:
\begin{verbatim}
  ! video/x-raw, width=640, height=480 !
\end{verbatim}
Legt für das Video den Pakettyp auf RTP und das Encoding auf VP8 fest.
\begin{verbatim}
  ! application/x-rtp, encoding-name=VP8 !
\end{verbatim}

Weitere Bestandteile der Pipeline
\begin{itemize}
\item \textbf{Muxers/Demuxers} – These elements encapsulate (pair) video and audio in a common container. Common formats are mp4, webm, ogg, mov.
E.g.:  mp4mux/qtdemux, webmmux/matroskademux.
\item \textbf{Payers/Depayers} – These elements prepare (payload) data prior and after it is transported over the Internet.
E.g.: rptvp8pay/rtpvp8depay,  rtph264/rtph264depay \textbf{*** hier raw...***}
\item \textbf{Converters} – These elements perform data manipulations like rotation, color change, modulation and cropping.
E.g.: audioconvert, audioresample, videoconvert, videoscale.
\end{itemize}

%%%%%%%%%%%%%%%%%%%%%%%%%%%%%%%%%%%%%%%%%%%%%%%%%%%%%%%%%%%
\subsection{Gstreamer Tests für Video \& Audio}

\textbf{Synthetische Quellen}

Video Test Source zum Display
\begin{verbatim}
Testbild:
  gst-launch-1.0 videotestsrc pattern=1 ! videoconvert ! autovideosink
Webcam oder intergrierte Kamera: 
  gst-launch-1.0 autovideosrc device=/dev/video0 ! autovideosink
\end{verbatim}

Audio Test Source zum Lautsprecher
\begin{verbatim}
  gst-launch-1.0 audiotestsrc ! audioconvert ! autoaudiosink
\end{verbatim}

Audio Test Quelle zu Fake Sink, aber trotzdem vollständige Pipeline
\begin{verbatim}
  gst-launch-1.0 audiotestsrc ! audioconvert ! fakesink
\end{verbatim}

Video Test Broadcast über TCP/HTTP
\begin{verbatim}
Sender
  gst-launch-1.0 videotestsrc horizontal-speed=5  ! vp8enc ! gdppay ! \
    tcpserversink host=127.0.0.1 port=5200
Empfänger
  gst-launch-1.0 -v tcpclientsrc port=5200 ! gdpdepay ! vp8dec ! \
    videoconvert ! autovideosink
\end{verbatim}

\textbf{Video Broadcast über RTP (via UDP) von der Webcam}
\begin{verbatim}
Sender
  gst-launch-1.0 v4l2src ! videoconvert ! video/x-raw, width=640,height=480 ! \
    omxh264enc ! rtph264pay pt=96 config-interval=1 ! \
    udpsink host=192.168.2.106 port =8554
Empfänger
  gst-launch-1.0 udpsrc port=8554 caps="application/x-rtp,media=video,\
    clockrate=90000,payload=96,encoding-name=H264" ! \
    rtph264depay ! avdec_h264 ! autovideosink
\end{verbatim}

\textbf{Video mit synthetischem Audio}
\begin{verbatim}
Sender:
  gst-launch-1.0 -v audiotestsrc ! audioconvert ! \
    audio/x-raw,channels=1,depth=16,width=16,rate=44100 ! rtpL16pay pt=97 ! \
    udpsink host=192.168.178.29 port=5001 v4l2src ! videoconvert ! \
    video/x-raw, width=640,height=480 ! omxh264enc ! \
    rtph264pay pt=96 config-interval=1 ! \
    udpsink host=192.168.178.29 port=5000
Empfänger
  gst-launch-1.0 udpsrc port=5001 ! application/x-rtp, clock-rate=44100, payload=97 ! 
    rtpL16depay ! audioconvert ! alsasink sync=false udpsrc port=5000 
    caps="application/x-rtp,media=video,payload=96,encoding-name=H264" ! rtph264depay !     
    avdec_h264 ! autovideosink
\end{verbatim}

\subsection{Kamera \& Mikrofon Streaming}
Erster Kamera Test:
\begin{verbatim}
  gst-launch-1.0 v4l2src ! xvimagesink
  gst-launch-1.0 v4l2src ! jpegdec ! xvimagesink
\end{verbatim}

\textbf{Video und Audio einer Webcam über RTP (in UDP Paketen)}
\begin{verbatim}
Sender
  gst-launch-1.0 -v alsasrc device=plughw:CARD=StudioTM,DEV=0 ! \
    audioconvert ! audio/x-raw,channels=1,depth=16,width=16,rate=44100 ! \
    rtpL16pay pt=97 ! udpsink host=192.168.178.29 port=5001 v4l2src ! \
    videoconvert ! video/x-raw, width=640,height=480 ! omxh264enc ! \
    rtph264pay pt=96 config-interval=1 ! udpsink host=192.168.178.29 port=5000
Empfänger
  gst-launch-1.0 udpsrc port=5001 ! application/x-rtp, clock-rate=44100,\
    payload=97 ! rtpL16depay ! audioconvert ! alsasink sync=false udpsrc \
    port=5000 caps="application/x-rtp,media=video,payload=96,encoding-name=H264" ! \
    rtph264depay ! avdec_h264 ! autovideosink
\end{verbatim}

\textbf{Raspi WebCam C920 Video/Audio an Webserver via RTP (UDP)}
\begin{verbatim}
Sender
  gst-launch-1.0 -v alsasrc device=plughw:1,0 ! audioconvert \
    ! audio/x-raw,channels=1,depth=16,width=16,rate=44100 \
    ! rtpL16pay pt=97 ! udpsink host=85.214.211.169 port=5001 v4l2src \
    ! videoconvert ! video/x-raw, width=640,height=480 ! omxh264enc \
    ! rtph264pay pt=96 config-interval=1 ! udpsink host=85.214.211.169 port=5000
\end{verbatim}

%%%%%%%%%%%%%%%%%%%%%%%%%%%%%%%%%%%%%%%%%%%%%%%%%%%%%%%%%%%%%%%%%%%%%%%%%%
\section{Janus-Gateway}
Was issn das????

\subsection{Installation von Janus}
\begin{verbatim}
sudo apt install slibmicrohttpd-dev libjansson-dev libnice-dev \
	libssl-dev libsrtp-dev libsofia-sip-ua-dev libglib2.0-dev \
	libopus-dev libogg-dev libcurl4-openssl-dev liblua5.3-dev \
	pkg-config gengetopt libtool automake

git clone https://github.com/meetecho/janus-gateway.git
cd janus-gateway
sh autogen.sh

./configure --prefix=/opt/janus

sudo make 
sudo make install 
make configs

wenn ein libsrtp Fehler angezeigt wird libsrtp aus dem Quelen bauen:

wget https://github.com/cisco/libsrtp/archive/v2.0.0.tar.gz
tar xfv v2.0.0.tar.gz
cd libsrtp-2.0.0
./configure --prefix=/usr --enable-openssl
make shared_library && sudo make install
\end{verbatim}

%%%%%%%%%%%%%%%%%%%%%%%%%%%%%%%%%%%%%%%%%%%%%%%%%%%%%%%%%%%%%%%%%%%%%%%%%%%%%
\subsection{Konfiguration}
Zum Streaming via gstreamer RTP muss noch die cfg Datei angepasst werden.\\
Alle cfg Dateien liegen in: /opt/janus/etc/janus\\

In Datei:\\
janus.transport.http.cfg
\begin{verbatim}
[general]
http = yes
ip = ...server-addr-ip-hostname
[admin]
admin_http = yes
\end{verbatim}

**** replace with working configs funkst noch nicht\\
janus.plugin.streaming.cfg
\begin{verbatim}
[gst-rpwc]
type = rtp 
id = 1 
description = RPWC H264 test streaming 
audio = yes 
audioport = 8005 
audiopt = 10 
audiortpmap = opus/48000/2 
video = yes 
videoport = 8004 
videopt = 96 
videortpmap = H264/90000 
videofmtp = profile-level-id=42e028\;packetization-mode=1 
\end{verbatim}

%%%%%%%%%%%%%%%%%%%%%%%%%%%%%%%%%%%%%%%%%%%%%%%%%%%%%%%%%%%%%%%%%%%%%%%%%%%%%
\subsection{nginx Webserver}
Installation von nginx Webserver zum Webseiten Testen in lokalem Netzwerk\\
Frei nach: https://www.digitalocean.com/community/tutorials/how-to-install-nginx-on-ubuntu-16-04
\begin{verbatim}
  sudo apt-get update
  sudo apt-get install nginx
  sudo ufw app list
\end{verbatim}

You should get a listing of the application profiles:\\
Terminal Ausgabe
\begin{verbatim}
Available applications:
  Nginx Full
  Nginx HTTP
  Nginx HTTPS
  OpenSSH
\end{verbatim}

Es ist meistens nötig Traffic auf dem Port 80 zu erlauben.\\
Man kann es konfigurieren mittels:
\begin{verbatim}
  sudo ufw allow 'Nginx HTTP'
  sudo ufw status
  http://server_domain_or_IP
\end{verbatim}

nginx Beispiele von janus nach /usr/share/nginx/html/demos kopieren
\begin{verbatim}
  sudo cp -r /opt/janus/share/janus/demos /usr/share/nginx/html/
\end{verbatim}
Im Browser unter http://ip-server/demos
wird trotzdem nur die Platzhalter Webseite angezeigt.

Um auf die kopierten Janus Webseiten zuzugreifen, muss noch etwas in der nginx Konfiguration geändert werden: /etc/nginx/sites-available
\begin{verbatim}
 #root /var/www/html;
 root /usr/share/nginx/html;
\end{verbatim}
Nun können die  Janus Beispiel html Dateien geladen werden.\\
http://192.168.178.25/demos/\\

Der Admin Monitor ist nur nach Eingabe eine Passwort verfügbar...\\
Entweder das Passwort aus /opt/janus/etc/janus/janus.cfg verwenden (admin\_secret = janusoverlord) oder einfach wie folgt anpassen:
\begin{verbatim}
/usr/share/nginx/html/demos/admin.js
line 14
var secret = "janusoverlord";
\end{verbatim}
Danach ist das Passwort schon voreingestellt und wird nicht mehr abgefragt.\\
Nun kann der Admin Bereich im Browser geöffnet werden:
\begin{verbatim}
http://ip-server/demos
\end{verbatim}

%%%%%%%%%%%%%%%%%%%%%%%%%%%%%%%%%%%%%%%%%%%%%%%%%%%%%%%%%%%%%%%%%%%%%%%%%%%%%
\subsection{gstreamer an janus}
Gstreamer Kommandos, um vom Raspberry V/A an das Janus-Gateway zu streamen\\
Zwei verschiedene Webcams wurden gestetet. \\
(1) Webcam HD Pro: CARD=C920,DEV=0 \\
(2) Microsoft® LifeCam Studio(TM): Webcam CARD=StudioTM,DEV=0 \\
Webcam (2) unterstützt h264parse nicht. Die Pipeline wurde alterntiv mit 
omxh264enc ergänzt. 
\begin{verbatim}
Webcam HD Pro: CARD=C920,DEV=0
  gst-launch-1.0 -v v4l2src ! h264parse ! rtph264pay config-interval=1 pt=96 ! \
    udpsink host=192.168.178.25 port=8004 alsasrc device=plughw:1,0 ! \ 
    audioconvert ! audioresample ! opusenc ! rtpopuspay ! udpsink \ 
    host=192.168.178.25 port=8005
 
Microsoft® LifeCam Studio(TM): Webcam CARD=StudioTM,DEV=0
Mit Bildskalierung, aber langsames Streamig (häufige Aussetzer)
  gst-launch-1.0 v4l2src ! videoconvert ! video/x-raw, width=640,height=480 !\
    omxh264enc ! rtph264pay pt=96 config-interval=1 ! \ 
    udpsink host=192.168.178.29 port=8004 alsasrc device=plughw:1,0 ! \
    audioconvert ! audioresample ! opusenc ! rtpopuspay ! udpsink \
    host=192.168.178.29 port=8005 
   
Microsoft® LifeCam Studio(TM): Webcam CARD=StudioTM,DEV=0
Ohne Bildskalierung, trotzdem schnelleres Streaming
  gst-launch-1.0 v4l2src ! omxh264enc ! rtph264pay pt=96 config-interval=1 ! \ 
    udpsink host=192.168.178.25 port=8006 alsasrc device=plughw:1,0 ! \
    audioconvert ! audioresample ! opusenc ! rtpopuspay ! udpsink \
    host=192.168.178.25 port=8007 
\end{verbatim}

%%%%%%%%%%%%%%%%%%%%%%%%%%%%%%%%%%%%%%%%%%%%%%%%%%%%%%%%%%%%%%%%%%%%%%%%%%