\section{Projektbeschreibung}

Beschreibung des Projektes...


\begin{minipage}{\textwidth}
    \begin{center}
        Caption for image
        \includegraphics[scale=0.4]{img/schemaproj.png} 
    \end{center}
\end{minipage}


Beschreibung der Schritte zum erreichen des Projektes
...

\section{ffmpeg, ffserver, ffplay}

\textbf{Linux / Raspian ffmpeg} %and pi\_jessie\_motion}

\begin{enumerate}
	\item remove (still needed?):\\
	sudo apt-get remove libavcodec-extra-56 libavformat56 libavresample2 libavutil54
	\item download ffmpeg precompiled armhf.deb:\\
	wget https://github.com/ccrisan/motioneye/wiki/precompiled/ffmpeg\_3.1.1-1\_armhf.deb\\
	sudo dpkg -i ffmpeg\_3.1.1-1\_armhf.deb
	%\item install:\\
	%sudo apt-get install curl libssl-dev libcurl4-openssl-dev libjpeg-dev libx264-142     
	%libavcodec56 libavformat56 libmysqlclient18 libswscale3 libpq5
	%\item install motion:\\
	%wget https://github.com/Motion-Project/motion/releases/download/release-4.0.1/pi\_jessie\_motion\_4.0.1-1\_armhf.deb\\
	%sudo dpkg -i pi\_jessie\_motion\_4.0.1-1\_armhf.deb
	
\textbf{Aufzeichnen von mp4 mit Audio}\\
ffmpeg -y -f video4linux2 -s 320x240 -i /dev/video0 -f alsa -i pulse -ac 2 -strict experimental ffmpegFile.mp4\\
Abspielen mit mplayer, ffplay oder vlc möglich.
	
	
\end{enumerate}


\section{Motion Installation \& Test}

Motion ist ein Programm, das in der Lage ist zu erkennen, wenn ein signifikanter Teil des Kamerabildes sich verändert. Es kann also 
Bewegung erkennen und einen Warnton übertragen. Kamera streaming 
Service welches verwendet werden kann, um den Videostream 
einer Webcam an eine IP Adresse zu leiten. Motion kann mit 
vielen Geräten verwendet werden. Unterstützt werden:
\begin{itemize}
\item V4L2 Webcams (closed source)
\item Video Frame Grabber
\item Network Kameras via HTTP, RTSP, RTMP
\item PI Kameramodul
\item Webcam
\end{itemize}

Video Stream zur IP Adresse des Devices (Raspi) im es im lokalen 
Netzwerk um Browser anzuzeigen....TODO

OP: Raspian\\
Setup: Streaming server motion\\

Anleitung nach Tutorial mit Anpassungen:\\
https://pimylifeup.com/raspberry-pi-webcam-server/\\

\textbf{Jessie and Strech are two debian major release}\\
Debian 9 (stretch) — current stable release\\
Debian 8 (jessie) — obsolete stable release\\

\textbf{Raspian pi\_streach\_motion}

\begin{enumerate}
	\item install:\\
	sudo apt-get install libmariadbclient18 libpq5 libavcodec57  libavformat57 libavutil55 libswscale4\\
	einige Pakete sind outdated und müssen durch aktuelle ersetzt werden:\\
	sudo apt install libx264-148\\
	libavcodec57\\
	libavformat57\\
	libmariadbclient-dev-compat\\
	default-libmysqlclient-dev\\
	libswscale

	\item download motion stretch deb\\
	sudo wget https://github.com/Motion-Project/motion/releases/download/release-4.0.1/pi\_stretch\_motion\_4.0.1-1\_armhf.deb\\
	sudo dpkg -i pi\_stretch\_motion\_4.0.1-1\_armhf.deb\\

	Configuring Motion:\\
	sudo vim /etc/motion/motion.conf\\
	daemon on\\
	stream\_localhost off\\
	if problems with freezing if motion occures\\
	output\_pictures off\\
	ffmpeg\_output\_movies off\\
	optional\\
	stream\_maxrate 100\\
	framerate 100\\
	width 640\\
	height 480

	\item setup daemon\\
	sudo vim /etc/default/motion\\
	start\_motion\_daemon=yes
\end{enumerate}

start stop motion and streaming by:\\
sudo service motion start\\
sudo service motion stop\\

check browser in local network, xxx ip adress of raspi (ip addr show):\\
192.168.1.xxx:8081

How to test if video and avi works at all:\\
Test raspi video codex and sound from avi video\\
omxplayer -p -o local dolbycanyon.avi\\
-o local = headphone jack

\section{Audio}
Um Audio auf den Headphone Jack umzuleiten (default ist HDMI) zuerst pulseaudio deinstallieren.\\
sudo apt remove pulseaudio\\

Dann die richtige Zuordnung setzen, 1 steht für local audio jack:\\
amixer cset numid=3 1
amixer cset numid=2 1\\
oder \\
amixer -c 0 cset numid=3 1\\

Audio Tests:\\
aplay /usr/share/scratch/Media/Sounds/Vocals/Singer1.wav\\
Facebook Video Call: OK (schlechter Sound)\\
Musikvidoes auf YoutTube: OK (guter Sound)\\

Kommandozeile, um alle Audiogeräte anzuzeigen:\\
pacmd list-sources

Listet alle *.ogg Audiodateien auf dem ubuntu Rechner auf:\\
pacmd list-samples\\
Lautsprechertest (abspielen):\\
mplayer /usr/share/sounds/ubuntu/stereo/button-pressed.ogg 


\section{RTMP versus WebRTC}

RTMP/RTSP are better for livestreaming in ‘live’ meaning. We can easily reduce the latency of RTMP or RTSP to around 1 second with just some simple setup and a good connection to the server, many streaming apps are using RTMP protocol nowaday. The only and biggest weakness of RTMP is it requires Flash and is not easily to be integrated into a website\\

WebRTC is something called the future for livestreaming, it is a peer-to-peer protocol which can reach ‘realtime’ latency for livestreaming(under 1 second). The weakness of WebRTC is it is hard when we need scaling, currently it is just approriate livestreaming in online meeting which require a small value of peers. There are many solution to overcome this, such as a hybrid solution combining WebRTC for input and RTMP/HLS/DASH for output.
