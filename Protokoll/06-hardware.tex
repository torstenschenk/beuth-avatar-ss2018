\section{Hardware}

%%%%%%%%%%%%%%%%%%%%%%%%%%%%%%%%%%%%%%%%%%%%%%%%%%%%%%%%%%%%%%%%%%%%%%%%%%%%%%%%%
\subsection{Raspberry Pi 3B+} \label{RefRaspi}
\textbf{Login via ssh}\\
Passwort und Benutzername können bereits bei der OP-Installation gesetzt werden. IP-Adresse kann über ip addr show abgefragt werden. Bei häufigen Verbindungen wird empfohlen eine statische IP-Adresse zu vergeben.\\

Bitte zuerst überprüfen, ob der ssh.service läuft.
\begin{verbatim}
/etc/init.d/ssh status
\end{verbatim}
Sollte eine Fehlermeldung erscheinen bitte ssh-server installieren.
\begin{verbatim}
sudo opt install openssh-server
\end{verbatim}
Falls nicht gestartet, in der Raspberry Konfiguration aktivieren:
\begin{verbatim}
  sudo raspi-config
\end{verbatim}
Danach ist es möglich über ssh einzuloggen.
\begin{verbatim}
  ssh pi@192.168.1.3
  pwd: xxxxxxxxxx (pwd vom Provider)
\end{verbatim}

\textbf{Raspberry Touchscreen Anzeige per Software Drehen}
Wenn das Raspberry 7\grqq{} Display ins Gehäuse eingebaut wird ist die 
Visualisierung des Desktops 180 Grad verdreht. Es müssen Bildschirmanzeige 
und Toucherkennung gedreht werden. Softwaretechnisch sind dies zwei verschiedene 
Dinge.\\

\textbf{RASPIAN OP}\\
Display \& Touchscreen können mit einem Befehl rotiert werden, 
bitte in /boot/config.txt eintragen:
\begin{verbatim}
  lcd_rotate=2
\end{verbatim}

\textbf{Andere OP}\\
Hier wird mittels des lcd\_display Befehls nur der Touchbildschrim 
gedreht. Es kann xrandr verwendet werden um zusätzlich die visuelle Darstellung zu 
um 180 Grad zu drehen. Display Infos \& drehen:
\begin{verbatim}
  xrandr -q
  xrandr --output HDMI-1 --rotate inverted
\end{verbatim}


%%%%%%%%%%%%%%%%%%%%%%%%%%%%%%%%%%%%%%%%%%%%%%%%%%%%%%%%%%%%%%%%%%%%%%%%%%%%%%%%%
\subsection{Strato Web Server}
\textbf{Login via ssh}
\begin{verbatim}
  ssh -X root@85.214.211.169
  ssh -X root@85.214.211.169 -L 5901:localhost:5901
  pwd: xxxxxxxxxx (pwd vom Provider)
\end{verbatim}
\textbf{Remote Desktop}\\
tightvncserver: server\\
xtightvncviewer: viewer
\begin{verbatim}sudo apt install tightvncserver xtightvncviewer
  set xtightvnciewer pwd
\end{verbatim}

\textbf{Full Login}
\begin{verbatim}
  ssh -X root@85.214.211.169 -L 5901:localhost:5901
  ssh Passwort eingeben
\end{verbatim}
Start vncserver
\begin{verbatim}
  vncserver :1
  echo "$DISPLAY"
\end{verbatim}
in server ssh console
\begin{verbatim}
  xtightvncviewer 127.0.0.1:1
  vnc Passwort eingeben
  Sollte eine Fehlermeldung erscheinen, mit htop
  F3 (search) htop suchen 
  ESC
  F9 (kill)
  2 SIGINT oder 9 SIGKILL
  und vnc Server nochmal starten
\end{verbatim}
X Fenster sollte sich öffnen\\

Dateien auf Server übertragen:
\begin{verbatim}
  scp <source> <destination>
  To copy a file from B to A while logged into B:
  scp /path/to/file username@a:/path/to/destination
  To copy a file from B to A while logged into A:
  scp username@b:/path/to/file /path/to/destination
\end{verbatim}
Beispiele:
\begin{verbatim}
Herunterladen aus der ssh Konsole: logged in PC -> eigener PC
  scp local_file_logged_in_pc user@my_own_host:dir_of_new_file_location
z.B. Herunterladen auf eigenen PC:
  scp /home/pi/Software/test/output.avi ts@192.168.178.25:/home/ts/Documents/tmp/
Hochladen aus lokaler Konsole: PC -> remote PC (raspberry)
  scp path_to_my_local_file pi@pi_host:dir_on_remote_pc
\end{verbatim}


%%%%%%%%%%%%%%%%%%%%%%%%%%%%%%%%%%%%%%%%%%%%%%%%%%%%%%%%%%%%%%%%%%%%%%%%%%%%%%%%%
\subsection{Audio Lautsprecher} \label{RefLautsprecher}
%Um Audio auf den Headphone Jack umzuleiten (default ist HDMI) zuerst pulseaudio deinstallieren.\\
%sudo apt remove pulseaudio\\

Die richtige Zuordnung setzen, sonst gibt es nur Sound vom 
abspielen von Audiodateien aber nicht im Browser. Bitte im 
Browser, z.B. mit YouTube. '1' steht für local audio jack über 
den den Lautsprecher angeschlossen ist:
\begin{verbatim}
  amixer cset numid=3 1
  amixer cset numid=2 1
  oder
  amixer -c 0 cset numid=3 1
\end{verbatim}

\textbf{Audio Tests}
\begin{verbatim}
  aplay /usr/share/scratch/Media/Sounds/Vocals/Singer1.wav
\end{verbatim}
Facebook Video Call: OK (schlechter Sound)\\
Musikvidoes auf YoutTube: OK (guter Sound)\\

Kommandozeile, um alle Audiogeräte anzuzeigen:\\
pacmd list-sources

Listet alle *.ogg Audiodateien auf dem ubuntu Rechner auf:\\
pacmd list-samples\\
Lautsprechertest (abspielen):\\
mplayer /usr/share/sounds/ubuntu/stereo/button-pressed.ogg 

%%%%%%%%%%%%%%%%%%%%%%%%%%%%%%%%%%%%%%%%%%%%%%%%%%%%%%%%%%%%%%%%%%%%%%%%%%%%%%%%%%%%
\subsection{Mikrofon der Webcam}
Anzeigen, ob Webcam erkannt wurde.\\
pacmd list-sources\\ 

Audio Stream aufnehmen:
\begin{verbatim}
  arecord -D plughw:1,0 -f cd test.wav
\end{verbatim}
abspielen mit:
\begin{verbatim}
  aplay test.wav
\end{verbatim}

Falls Hardwarekennung unbekannt, zeige alle Geräte:
\begin{verbatim}
  arecord -L
\end{verbatim}
Interessant ist der letzte Eintrag plughw, diesen kann man direkt für
ffmpeg verwenden, z.B.:
\begin{verbatim}
  webcam: ...plughw:CARD=C920,DEV=0 ...
  usb-micro: arecord -D plughw:CARD=Device,DEV=0 -f cd test.wav
\end{verbatim}

Wenn beim Testen mit ffmpeg der Fehler: 
\textbf{Unknown input format: 'alsa'}\\ 
auftreten, muss ffmpeg aus den Quellen kompiliert werden, siehe 
Softwarekapitel oder\\ 
https://lb.raspberrypi.org/forums/viewtopic.php?t=205181

