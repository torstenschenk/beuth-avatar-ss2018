\newpage
\section{Software}

%%%%%%%%%%%%%%%%%%%%%%%%%%%%%%%%%%%%%%%%%%%%%%%%%%%%%%%%%%%%%%%%%%%%%%%%%%
\subsection{Raspberry Touchscreen Anzeige per Software Drehen}
Wenn das Raspberry 7\grqq{} Display ins Gehäuse eingebaut wird ist die 
Visualisierung des Desktops 180 Grad verdreht. Es müssen Bildschirmanzeige 
und Toucherkennung gedreht werden. Softwaretechnisch sind dies zwei verschiedene 
Dinge.\\

\textbf{RASPIAN OP}\\
Display \& Touchscreen können mit einem Befehl rotiert werden, 
bitte in /boot/config.txt eintragen:
\begin{verbatim}lcd_rotate=2\end{verbatim}

\textbf{Andere OP}\\
Hier wird mittels des lcd\_display Befehls nur der Touchbildschrim 
gedreht. Es kann xrandr verwendet werden um zusätzlich die visuelle Darstellung zu 
um 180 Grad zu drehen. Display Infos \& drehen:
\begin{verbatim}
xrandr -q
xrandr --output HDMI-1 --rotate inverted
\end{verbatim}


%%%%%%%%%%%%%%%%%%%%%%%%%%%%%%%%%%%%%%%%%%%%%%%%%%%%%%%%%%%%%%%%%%%%%%%%%
\subsection{ffmpeg, ffserver, ffplay}

\textbf{Linux / Raspian ffmpeg mit alsa} %and pi\_jessie\_motion}
   
REWORK in GERMAN:\\
No, unfortunately not. You must recompile ffmpeg to add enable additional libraries. Below is the script I build to compile ffmpeg with alsa, fdk-aac, and libx264 support. It will install ffmpeg in your home folder inside a "ffmpeg" folder, so you'll need to call it specifically from there unless you add it to your path. I recommend uninstalling your current ffmpeg before using my script.

Btw, I am able to stream directly to YouTube now without any issues. I use an external USB sound card and the PiCam v2 and stream a 1920x1080@25fps video stream with a 192kbps stereo audio stream mixed in. It works great!\\

Wichtige Info, ffserver wurde beim Upgrade auf ffmpeg Version 4.0 gelöscht. Letzte Version mit ffserver ist 3.4. Daher git checkout 3.4!

Installation von FFmpeg aus Source-Code: 
\begin{verbatim}
#!/bin/bash
# My first script
# 
# to install ffserver, you have to check out the 3.5 of 3.4 version of ffmepg git repo
# to be able to compile ffmpeg with x264, you also have to choose an old version from git repo
# tested was the version, before x264_bit_depth was removed (needed for ffmpeg 3.4, 3.5)
# x264 hash key checkout: 7839a9e1f03b49e3e0cbfcb3091093af7c6d54ee
#
wget http://downloads.xiph.org/releases/vorbis/libvorbis-1.3.3.tar.gz
tar -xf libvorbis-1.3.3.tar.gz
cd libvorbis-1.3.3/
./configure --host=arm-unknown-linux-gnueabi --enable-static
make
sudo make install
cd ..
# libogg
wget http://downloads.xiph.org/releases/ogg/libogg-1.3.1.tar.gz
tar -xf libogg-1.3.1.tar.gz
cd libogg-1.3.1/
./configure --host=arm-unknown-linux-gnueabi --enable-static
make
sudo make install
cd ..
# libtheora
wget http://downloads.xiph.org/releases/theora/libtheora-1.1.1.tar.bz2
tar -xf libtheora-1.1.1.tar.bz2
cd libtheora-1.1.1/
./configure --host=arm-unknown-linux-gnueabi --enable-static
make
sudo make install
cd ..
git clone http://git.videolan.org/git/x264.git
cd x264
./configure --host=arm-unknown-linux-gnueabi --enable-static --disable-opencl
echo "Compiling x264"
make
sudo make install
cd ..
# extra alsa
wget ftp://ftp.alsa-project.org/pub/lib/alsa-lib-1.1.1.tar.bz2
tar xjf alsa-lib-1.1.1.tar.bz2
cd alsa-lib-1.1.1/
./configure --host=arm-unknown-linux-gnueabi --enable-static
make -j4
sudo make install
cd ..
# libvpx
git clone https://chromium.googlesource.com/webm/libvpx
cd libvpx
./configure --enable-static
make -j4
sudo make install
cd ..
# libsdl
wget http://www.libsdl.org/release/SDL-1.2.15.tar.gz
tar xzvf SDL-1.2.15.tar.gz
cd SDL-1.2.15
./configure --host=arm-unknown-linux-gnueabi --enable-static
make -j4
sudo make install
cd ..
git clone https://github.com/FFmpeg/FFmpeg.git
cd ffmpeg
./configure --arch=armel --target-os=linux --enable-gpl --enable-libx264 --enable-nonfree --enable-libtheora --enable-libvorbis --enable-libvpx
make
sudo make install
\end{verbatim}	

Als erste Tests kann ffmpeg dazu verwendet werden Videos mit oder ohne Audio zu speichern.\\
Dabei traten beim .ogg Format ungewöhnliche Speed-Ups oder Sprünge statt. Mp4 ist von 
ausgezeichneter Qualität. Tests ohne Audio:
\begin{verbatim}
ffmpeg -f v4l2 -r 25 -s 640x480 -i /dev/video0 out.avi
ffmpeg -f v4l2 -r 25 -s 640x480 -i /dev/video0 out.mp4
ffmpeg -f v4l2 -r 25 -s 640x480 -i /dev/video0 out.ogg
\end{verbatim}

Test mit Audio:
\begin{verbatim}
arecord -D plughw:1,0 -f cd test.wav
mplayer test.wav
\end{verbatim}


Danach kann über den ffserver auf eine Webpage gestreamt werden:\\
ffserver.config
\begin{verbatim}
HTTPPort 9090
HTTPBindAddress 0.0.0.0
MaxHTTPConnections 2000
MaxClients 1000
MaxBandwidth 100000
#NoDaemon

<Feed feed1.ffm>
        File /tmp/feed1.ffm
        FileMaxSize 200K
        ACL allow 127.0.0.1
</Feed>

<Stream test.ogg>
        Format ogg
        Feed feed1.ffm

        VideoCodec libtheora
        VideoFrameRate 24
        VideoBitRate 512
        VideoSize 320x240
        VideoQMin 1
        VideoQMax 31
        VideoGopSize 12
        PreRoll 0
        AVOptionVideo flags +global_header
        Noaudio
</Stream>

<Stream stat.html>
        Format status
        # Only allow local people to get the status
        ACL allow localhost
        ACL allow 192.168.0.0 192.168.255.255
</Stream>                         
\end{verbatim}

Start ffserver und ffmpeg:
\begin{verbatim}
ffserver -f ffserver.config
ffmpeg -f v4l2 -i /dev/video0 -vcodec libtheora http://localhost:9090/feed1.ffm
\end{verbatim}
	
% Aufzeichnen von wav mit ffmpeg:
% \begin{verbatim}./ffmpeg -f alsa -i hw:1 alsaout.wav \end{verbatim}
% abspielen:
% \begin{verbatim}aplay alsaout.wav \end{verbatim}
% Aufzeichnen von webm mit ffmpeg:
% \begin{verbatim}
% ./ffmpeg -f alsa -i hw:1 -strict experimental alsaout.webm  
% ACHTUNG: bei ffmpeg dürfen keine Anführungszeichen um plughw gesetzt werden!
% webcam: ./ffmpeg -f alsa -i plughw:CARD=C920,DEV=0 -strict experimental alsaout.webm
%
% \end{verbatim}
% abspielen:
% \begin{verbatim}
%
% \end{verbatim}


%%%%%%%%%%%%%%%%%%%%%%%%%%%%%%%%%%%%%%%%%%%%%%%%%%%%%%%%%%%%%%%%%%%%%%%%%%%
\subsection{Motion Installation \& Test}

Motion ist ein Programm, das in der Lage ist zu erkennen, wenn ein signifikanter Teil des Kamerabildes sich verändert. Es kann also 
Bewegung erkennen und einen Warnton übertragen. Kamera streaming 
Service welches verwendet werden kann, um den Videostream 
einer Webcam an eine IP Adresse zu leiten. Motion kann mit 
vielen Geräten verwendet werden. Unterstützt werden:
\begin{itemize}
\item V4L2 Webcams (closed source)
\item Video Frame Grabber
\item Network Kameras via HTTP, RTSP, RTMP
\item PI Kameramodul
\item Webcam
\end{itemize}

Video Stream zur IP Adresse des Devices (Raspi) im es im lokalen 
Netzwerk um Browser anzuzeigen....TODO

OP: Raspian\\
Setup: Streaming server motion\\

Anleitung nach Tutorial mit Anpassungen:\\
https://pimylifeup.com/raspberry-pi-webcam-server/\\

\textbf{Jessie and Strech are two debian major release}\\
Debian 9 (stretch) — current stable release\\
Debian 8 (jessie) — obsolete stable release\\

\textbf{Raspian pi\_strech\_motion}

\begin{enumerate}
	\item install:\\
	sudo apt-get install libmariadbclient18 libpq5 libavcodec57  libavformat57 libavutil55 libswscale4\\
	einige Pakete sind outdated und müssen durch aktuelle ersetzt werden:\\
	sudo apt install libx264-148\\
	libavcodec57\\
	libavformat57\\
	libmariadbclient-dev-compat\\
	default-libmysqlclient-dev\\
	libswscale

	\item download motion stretch deb\\
	sudo wget https://github.com/Motion-Project/motion/releases/download/release-4.0.1/pi\\
	\_stretch\_motion\_4.0.1-1\_armhf.deb
	
	sudo dpkg -i pi\_stretch\_motion\_4.0.1-1\_armhf.deb\\

	Configuring Motion:\\
	sudo vim /etc/motion/motion.conf\\
	daemon on\\
	stream\_localhost off\\
	if problems with freezing if motion occures\\
	output\_pictures off\\
	ffmpeg\_output\_movies off\\
	optional\\
	stream\_maxrate 100\\
	framerate 100\\
	width 640\\
	height 480

	\item setup daemon\\
	sudo vim /etc/default/motion\\
	start\_motion\_daemon=yes
\end{enumerate}

start stop motion and streaming by:\\
sudo service motion start\\
sudo service motion stop\\

check browser in local network, xxx ip adress of raspi (ip addr show):\\
192.168.1.xxx:8081

How to test if video and avi works at all:\\
Test raspi video codex and sound from avi video\\
omxplayer -p -o local dolbycanyon.avi\\
-o local = headphone jack

%%%%%%%%%%%%%%%%%%%%%%%%%%%%%%%%%%%%%%%%%%%%%%%%%%%%%%%%%%%%%%%%%%%%%%%%%%
\subsection{gstreamer}

Wenn Fehler beim Compilieren eines gstream Testprogramms auftreten, z.B.
\begin{verbatim}
Package gstreamer-1.0 was not found in the pkg-config search path.
Perhaps you should add the directory containing `gstreamer-1.0.pc'
to the PKG_CONFIG_PATH environment variable
No package 'gstreamer-1.0' found
playback-tutorial-6.c:1:10: fatal error: gst/gst.h: No such file or directory
\end{verbatim}

gstreamer-1.0 ist der folgenden lib enthalten:\\
sudo apt install libgstreamer1.0-dev\\

\textbf{Beispiel Programme gstreamer kompilieren}
gcc playback-tutorial-6.c -o playback-tutorial-6 `pkg-config --cflags --libs gstreamer-1.0`

%%%%%%%%%%%%%%%%%%%%%%%%%%%%%%%%%%%%%%%%%%%%%%%%%%%%%%%%%%%%%%%%%%%%%%%%%%

