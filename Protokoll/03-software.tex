\newpage
\section{Software}

%%%%%%%%%%%%%%%%%%%%%%%%%%%%%%%%%%%%%%%%%%%%%%%%%%%%%%%%%%%%%%%%%%%%%%%%%%
\subsection{Raspberry Touchscreen Anzeige per Software Drehen}
Wenn das Raspberry 7\grqq{} Display ins Gehäuse eingebaut wird ist die 
Visualisierung des Desktops 180 Grad verdreht. Es müssen Bildschirmanzeige 
und Toucherkennung gedreht werden. Softwaretechnisch sind dies zwei verschiedene 
Dinge.

\# Zeige Namen aller angeschlossenen Displays\\
xrandr -q\\

\# 180 Grad drehen\\
xrandr --output HDMI-1 --rotate inverted\\

\# Touchscreen muss auch rotiert werden\\
\# bitte einfügen in /boot/config.txt:\\
\# rote touchscreen (display rotated via xrandr startup script)\\
lcd\_rotate=2\\

%%%%%%%%%%%%%%%%%%%%%%%%%%%%%%%%%%%%%%%%%%%%%%%%%%%%%%%%%%%%%%%%%%%%%%%%%
\subsection{ffmpeg, ffserver, ffplay}

\textbf{Linux / Raspian ffmpeg mit alsa} %and pi\_jessie\_motion}

\begin{enumerate}
%	\item remove (still needed?):\\
%	sudo apt-get remove libavcodec-extra-56 libavformat56 libavresample2 libavutil54
%	\item download ffmpeg precompiled armhf.deb:\\
%	wget https://github.com/ccrisan/motioneye/wiki/precompiled/ffmpeg\_3.1.1-1\_armhf.deb\\
%	sudo dpkg -i ffmpeg\_3.1.1-1\_armhf.deb

  \item Neue Build-Anweisung
	
  \item Aufzeichnen von mp4 mit Audio\\
  
  
  %ffmpeg -y -f video4linux2 -s 320x240 -i /dev/video0 -f alsa -i pulse -ac 2 -strict experimental ffmpegFile.mp4\\
  %Abspielen mit mplayer, ffplay oder vlc möglich.

\end{enumerate}

%%%%%%%%%%%%%%%%%%%%%%%%%%%%%%%%%%%%%%%%%%%%%%%%%%%%%%%%%%%%%%%%%%%%%%%%%%%
\subsection{Motion Installation \& Test}

Motion ist ein Programm, das in der Lage ist zu erkennen, wenn ein signifikanter Teil des Kamerabildes sich verändert. Es kann also 
Bewegung erkennen und einen Warnton übertragen. Kamera streaming 
Service welches verwendet werden kann, um den Videostream 
einer Webcam an eine IP Adresse zu leiten. Motion kann mit 
vielen Geräten verwendet werden. Unterstützt werden:
\begin{itemize}
\item V4L2 Webcams (closed source)
\item Video Frame Grabber
\item Network Kameras via HTTP, RTSP, RTMP
\item PI Kameramodul
\item Webcam
\end{itemize}

Video Stream zur IP Adresse des Devices (Raspi) im es im lokalen 
Netzwerk um Browser anzuzeigen....TODO

OP: Raspian\\
Setup: Streaming server motion\\

Anleitung nach Tutorial mit Anpassungen:\\
https://pimylifeup.com/raspberry-pi-webcam-server/\\

\textbf{Jessie and Strech are two debian major release}\\
Debian 9 (stretch) — current stable release\\
Debian 8 (jessie) — obsolete stable release\\

\textbf{Raspian pi\_strech\_motion}

\begin{enumerate}
	\item install:\\
	sudo apt-get install libmariadbclient18 libpq5 libavcodec57  libavformat57 libavutil55 libswscale4\\
	einige Pakete sind outdated und müssen durch aktuelle ersetzt werden:\\
	sudo apt install libx264-148\\
	libavcodec57\\
	libavformat57\\
	libmariadbclient-dev-compat\\
	default-libmysqlclient-dev\\
	libswscale

	\item download motion stretch deb\\
	sudo wget https://github.com/Motion-Project/motion/releases/download/release-4.0.1/pi\\
	\_stretch\_motion\_4.0.1-1\_armhf.deb
	
	sudo dpkg -i pi\_stretch\_motion\_4.0.1-1\_armhf.deb\\

	Configuring Motion:\\
	sudo vim /etc/motion/motion.conf\\
	daemon on\\
	stream\_localhost off\\
	if problems with freezing if motion occures\\
	output\_pictures off\\
	ffmpeg\_output\_movies off\\
	optional\\
	stream\_maxrate 100\\
	framerate 100\\
	width 640\\
	height 480

	\item setup daemon\\
	sudo vim /etc/default/motion\\
	start\_motion\_daemon=yes
\end{enumerate}

start stop motion and streaming by:\\
sudo service motion start\\
sudo service motion stop\\

check browser in local network, xxx ip adress of raspi (ip addr show):\\
192.168.1.xxx:8081

How to test if video and avi works at all:\\
Test raspi video codex and sound from avi video\\
omxplayer -p -o local dolbycanyon.avi\\
-o local = headphone jack

%%%%%%%%%%%%%%%%%%%%%%%%%%%%%%%%%%%%%%%%%%%%%%%%%%%%%%%%%%%%%%%%%%%%%%%%%%
\subsection{gstreamer}

Wenn Fehler beim Compilieren eines gstream Testprogramms auftreten, z.B.
\begin{verbatim}
Package gstreamer-1.0 was not found in the pkg-config search path.
Perhaps you should add the directory containing `gstreamer-1.0.pc'
to the PKG_CONFIG_PATH environment variable
No package 'gstreamer-1.0' found
playback-tutorial-6.c:1:10: fatal error: gst/gst.h: No such file or directory
\end{verbatim}

gstreamer-1.0 ist der folgenden lib enthalten:\\
sudo apt install libgstreamer1.0-dev\\

\textbf{Beispiel Programme gstreamer kompilieren}
gcc playback-tutorial-6.c -o playback-tutorial-6 `pkg-config --cflags --libs gstreamer-1.0`

%%%%%%%%%%%%%%%%%%%%%%%%%%%%%%%%%%%%%%%%%%%%%%%%%%%%%%%%%%%%%%%%%%%%%%%%%%

