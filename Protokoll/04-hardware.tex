\newpage
\section{Hardware}

%%%%%%%%%%%%%%%%%%%%%%%%%%%%%%%%%%%%%%%%%%%%%%%%%%%%%%%%%%%%%%%%%%%%%%%%%%%%%%%%%%%%%%%%%%%%%%%%%%%%
\subsection{Strato Web Server}
\textbf{Login via ssh}\\
ssh -X root@85.214.211.169\\
ssh -X root@85.214.211.169 -L 5901:localhost:5901\\
pwd: xxxxxxxxxx (pwd vom Provider)

\textbf{Remote Desktop}\\
tightvncserver: server\\
xtightvncviewer: viewer\\
sudo apt install tightvncserver xtightvncviewer\\
\# set xtightvnciewer pwd\\

\textbf{Full Login}\\
ssh -X root@85.214.211.169 -L 5901:localhost:5901\\
ssh Passwort eingeben\\
\# Start vncserver\\
vncserver :1\\
echo \grqq{}\$DISPLAY\grqq{}\\
\# in server ssh console\\
xtightvncviewer 127.0.0.1:1\\
vnc Passwort eingeben\\
\# X Fenster sollte sich öffnen

%%%%%%%%%%%%%%%%%%%%%%%%%%%%%%%%%%%%%%%%%%%%%%%%%%%%%%%%%%%%%%%%%%%%%%%%%%%%%%%%%%%%%%%%%%%%%%%%%%%%%
\subsection{Audio Lautsprecher}
Um Audio auf den Headphone Jack umzuleiten (default ist HDMI) zuerst pulseaudio deinstallieren.\\
sudo apt remove pulseaudio\\

Dann die richtige Zuordnung setzen, 1 steht für local audio jack:\\
amixer cset numid=3 1
amixer cset numid=2 1\\
oder \\
amixer -c 0 cset numid=3 1\\

Audio Tests:\\
aplay /usr/share/scratch/Media/Sounds/Vocals/Singer1.wav\\
Facebook Video Call: OK (schlechter Sound)\\
Musikvidoes auf YoutTube: OK (guter Sound)\\

Kommandozeile, um alle Audiogeräte anzuzeigen:\\
pacmd list-sources

Listet alle *.ogg Audiodateien auf dem ubuntu Rechner auf:\\
pacmd list-samples\\
Lautsprechertest (abspielen):\\
mplayer /usr/share/sounds/ubuntu/stereo/button-pressed.ogg 

