\section*{Aufgabe 4:}
Aliasing tritt immer dann auf, wenn der Raum einer Menge von Daten (z.B. durch Interpolation oder Resampling) verkleinert wird, sodass sich mehrere Daten nicht mehr voneinander unterschieden lassen.
Da der Marching-Cubes Algorithmus ausgehend von einer stetigen Funktion den $\mathbb{R}^3$ in diskrete Würfel aufteilt und die Funktion nur in den Ecken dieser Würfel auswertet, tritt auch hier Aliasing auf.